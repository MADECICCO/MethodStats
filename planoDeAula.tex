\documentclass[oneside,a4paper,12pt]{article}
\usepackage[english,brazilian]{babel}
\usepackage[alf]{abntex2cite}
\usepackage[utf8]{inputenc}
\usepackage[T1]{fontenc}
\usepackage[top=20mm, bottom=20mm, left=20mm, right=20mm]{geometry}%margens cima, baixo, esquerda direita
\usepackage{framed}
\usepackage{booktabs} %Pacote para deixar tabelas mais bonitas.
\usepackage{color} %Pacote de Cores
\usepackage{hyperref} %Pacotes para Hiperlinks
\usepackage{graphicx} %Pacote de imagens
\graphicspath{{./Figuras/}}%Direciona as imagens para uma pasta chamada "Figuras" (uso isso para organizar. Uma vez que todas as imagens vao ficar em uma pasta isolada)    
\definecolor{shadecolor}{rgb}{0.8,0.8,0.8}

%+++ CABECALHO E RODAPÉ

%--- CABECALHO E RODAPÉ



%FAZ EDICOES AQUI (somente no conteudo que esta entre entre as ultimas  chaves de cada linha!!!)
\newcommand{\universidade}{INMETRO}
\newcommand{\centro}{CENTRO Y}
\newcommand{\departamento}{POS GRADUACAO}
\newcommand{\curso}{........}
\newcommand{\professores}{Marcelo De Cicco}
\newcommand{\professoresg}{Gabriel}
\newcommand{\disciplina}{........}
\newcommand{\tema}{......}
\newcommand{\turma}{....}
\newcommand{\data}{\today}
\newcommand{\tempodeaula}{180 minutos}
\newcommand{\creditos}{15 aulas, 45 horas ao total, perfazendo 3 creditos}
%ATE AQUI !!!	

\begin{document}
	\pagestyle{empty}
	
	\begin{center}
		\includegraphics[width=\linewidth/5]{inmetro-logo.png}%LOGOTIPO DA INSTITUICAO
	 	\vspace{2pt} 	
		
		\universidade
		\par
		\centro
		\par
		\departamento
		\par
		Curso de \curso
		\par
		\vspace{12pt}
		\LARGE \textbf{Plano de Aula}
		
	\end{center}
	
	\vspace{12pt}
	
	\begin{tabular}{ |l|p{12cm}| }
		
		\hline
		\multicolumn{2}{|c|}{\textbf{Dados de Identificação}} \\
		\hline
		Professor:         &    \professores           \\
		\hline
		Professor:         &    \professoresg           \\
		\hline
		Disciplina:        &    \disciplina          \\
		\hline
		Tema:              &    \tema                \\
		\hline
		Turma:             &    \turma               \\
		\hline
		Data:              &    \data                \\
		\hline
		Duração da aula:   &    \tempodeaula         \\
		\hline
			Creditos:   &    \creditos         \\
		\hline
		
		
		
		
	\end{tabular}
	\vspace{12pt}
	
	\begin{snugshade}
		\section{Objetivos} % a serem alcançados pelos alunos e não pelo professor. Podem ser divididos em gerais e específicos. 
		
		\lefthead
Apos a conclusao do curso , o aluno devera estar apto a construir algoritmos basicos e plotar graficos de analises estatisticas e modelos.
        \begin{itemize}
        \item Como construir a logica de dados voltados a analise estatistica 
        \item Desenvolver habilidades de data minning, preparando dados brutos para analise .
        \item Construir \emph{pipelines} basicos voltados a automatizacao de processamento de dados.
        \item Realizar analise estatitica usando algoritmos em linguagem Python e R.
        \item Construir e formatar Graficos .
        \end{itemize} \vspace{0.3cm}
		
		
		
		
		
	\end{snugshade}
	
		\subsection{Geral} % projeta resultado geral relativo a execução de conteúdos e procedimentos.
		%TEXTO COM O OBJJETIVO GERAL
			Objetivo geral aqui...
	
		\subsection{Específicos} % especificam resultados esperados observáveis (geralmente de 3 a 4).
		%TEXTO COM OS OBJJETIVOS ESP
			\begin{itemize}
				\item Objetivo específico a;
				\item Objetivo específico b;
				\item Objetivo específico c. 
			\end{itemize}
	
	\begin{snugshade}
		\section{Ementas} % conteúdos programados para a aula organizados em tópicos (de 4 a 8).
	\end{snugshade}
		%TEXTO COM OS CONTEUDOS
		\begin{itemize}
		\item Parte I - Introducao e uso basico de Python para ferramentras estatisitcas:
		 \begin{itemize}
			\item Aula 1:\\
			Instalacao de pacote Python e R em sistemas Windows, Linux e Mac. Introducao a linguagem de programacao Python e R. Uso de Python Notebook e ambiente IDE;
			\item Aula 2:\\
			Estruturacao de dados, uso de Tuplas, listas, Arrays, Dictionaries e Data Frame. Indexacao e fatiamento de dados, vetores e arrays, Usando Notebooks e R;
			\item Aula 3:\\
			Programacao em Python, uso do pacote Pandas, estruturacao de dados para estatistica. Manipulacao dos dados, agrupamento, aplicando 'statmodel' e 'seaborn';
			\item Aula 4:\\
			Entrada de daods, lendo entradas, uso de 'regular expressions', importacao de arquivos em formato Excel e Matlab;
			\item Aula 5:\\
			Visualizando dados estatisticos, categorias e numericos. Plotagem em Python, dados univariados, bivariados e multivariados;
			
		\end{itemize}
		\item Parte II - Distribuicoes e testes de hipoteses:
		\begin{itemize}
		    \item Aula 6:\\
		    Estatistica basica, populacao e amostra, distribuicao de Probabilidades, discrteas e continuas. Valor Esperado e Variancia,graus de liberdade;
		    \item Aula 7:\\
		    Distribuicao de uma bivariavel, caracterizacao, distribuicao discretas, distribuicao normal, continua, outras distribuicoes;
		    \item Aula 8:\\
		    Conceitos de hipoteses, erros, p-valor; tamanho da amostra. Generalizacoes e aplicacoes. Tipos de erros, tamanho da amostra, sensitividade e especificidade;
		    \item Aula 9:\\
		    Testes de hipoteses, distribuicao de uma media amostral, comparacoes de 2 grupos. Testes estatisticos de hipoteses versus modelagem Estatistica. Comparacao de grupos multiplos. Selecionando o teste certo para comparacao de grupos;
		    \item Aula 10: \\
		    Testes de dados categorizados, intervalos de confianca, tabelas de frequencia, teste de confianca  , chi quadrado, teste  exato de Fisher, Teste McNemar, teste Q de Cochran;
		    
		    	\item Aula 11:\\
		    Analise de tempo de sobrevivencia, 'survival distribution', probabilidade de sobrevivencia;
		    
		\end{itemize}{}
		
		 \item Parte III - Modelagem
		\begin{itemize}
		   
	         \item Aula 12:\\
	         Modelagem estistica, aplicacao de modelos de regressao linear: correlacao linear. Regressao linear geral. Analise de regressao linear. O valor do ajuste $R^{2}$. Interpretando os coeficientes de modelagem;
	         
			\item Aula 13:\\
			Analise de residuos, outliers e premissas dos modelos de regressao linear, interpretacao dos resultados;
			
			\item Aula 14:\\
			Analise de dados mulitvariados, testes em dados discretos;
			\item Aula 15:\\
			Introducao a \textit{Machine Learning};
		
		\end{itemize}
		
	\end{itemize}
	
	\begin{snugshade}
		\section{Procedimentos metodológicos} % estratégias relevantes adotadas para alcançar os objetivos. 
	\end{snugshade}
	%TEXTO COM A METODOLOGIA DA AULA
	O procedimento...  
	
	\begin{snugshade}
		\section{Recursos didáticos} 
		
		% Textbook &  Software
\lefthead{ Notebook, \& Software}\\

\noindent Notebook: qualquer marca e sistema operacional\\

\noindent Software: SO -> Windows Ananconda (ultima versao)\\
\noindent Software: SO -> Linux Conda (ultima versao)\\
\noindent Software: SO -> IOS Ananconda ou Conda (ultima versao)\\

\noindent\\[0.3cm]

		
		
		
	\end{snugshade}
	%TEXTO COM OS RECURSOS UTILIZADOS NA AULA
		Utilizarei...
	
	\begin{snugshade}
		\section{Avaliação} % pode ser realizada com diferentes propósitos (diagnóstica, formativa e somativa). Interessante explicitar a atividade avaliativa e os critérios de correção.
	\end{snugshade}
	%TEXTO COM A DESCRICAO DO PROCESSO DE AVALIACAO
	Prof.Marcelo De Cicco:\\
	A Avaliação sera realizada atraves de questionarios a serem entregrues em data previamente acordada com a turma.\\
	\vspace{12pt} 
	Prof. Gabriel:\\
	
	\vspace{12pt}
	% Referências bibliográficas - como no plano de aula v. nao faz citacoes, as bibliografias nao podem ser geradas automaticamente. Entao tera que preencher manualmente   :(
	
	\begin{thebibliography}{}
	%EXEMPLO 1
	\bibitem{}
	\noindent Textbook: \emph{Statistics, Data Mining, and Machine Learning in Astronomy}, by Zeljko Ivezic, Andrew J. Connolly, Jacob T. Vanderplas & Alexandre Gray. \\
    \noindent Textbook: \emph{A student's guide to PYTHON for physical modeling}, by Jesse M. Kinder and Philip Nelson. \\

	

	%EXEMPLO 2
	\bibitem{}
	
	\end{thebibliography}
		
	
\end{document}
